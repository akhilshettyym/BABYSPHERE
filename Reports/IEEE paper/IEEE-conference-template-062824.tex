\documentclass[conference]{IEEEtran}
\IEEEoverridecommandlockouts
% The preceding line is only needed to identify funding in the first footnote. If that is unneeded, please comment it out.
%Template version as of 6/27/2024

\usepackage{cite}
\usepackage{amsmath,amssymb,amsfonts}
\usepackage{algorithmic}
\usepackage{graphicx}
\usepackage{textcomp}
\usepackage{xcolor}
\def\BibTeX{{\rm B\kern-.05em{\sc i\kern-.025em b}\kern-.08em
    T\kern-.1667em\lower.7ex\hbox{E}\kern-.125emX}}
\begin{document}

\title{CLOUD-BASED SMART MONITORING SYSTEM
FOR BABY HEALTH AND SAFETY
% {\footnotesize \textsuperscript{*}Note: Sub-titles are not captured for https://ieeexplore.ieee.org  and should not be used}
% \thanks{Identify applicable funding agency here. If none, delete this.}
}
\author{\IEEEauthorblockN{1\textsuperscript{st} Aaron Tauro}
\IEEEauthorblockA{\textit{Dept. of CSE} \\
\textit{St. Joseph Engineering College}\\
Mangaluru, India \\
aarontauro00@gmail.com}
\and
\IEEEauthorblockN{2\textsuperscript{nd} Abhik L Salian}
\IEEEauthorblockA{\textit{Dept. of CSE} \\
\textit{St. Joseph Engineering College}\\
Mangaluru, India \\
abhiksalian0728@gmail.com}
\and 
\IEEEauthorblockN{3\textsuperscript{rd} Akhil Shetty M}
\IEEEauthorblockA{\textit{Dept. of CSE} \\
\textit{St. Joseph Engineering College}\\
Mangaluru, India \\
akhilshettym2003@gmail.com}
\and
\IEEEauthorblockN{4\textsuperscript{th} H Karthik P Nayak}
\IEEEauthorblockA{\textit{Dept. of CSE} \\
\textit{St. Joseph Engineering College}\\
Mangaluru, India \\
karthiknayak2003@gmail.com}
\and
\IEEEauthorblockN{5\textsuperscript{th}  Nishayne E Vaz}
\IEEEauthorblockA{\textit{Dept. of CSE} \\
\textit{St. Joseph Engineering College}\\
Mangaluru, India \\
vaznish@gmail.com}
% \and
% \IEEEauthorblockN{6\textsuperscript{th} Given Name Surname}
% \IEEEauthorblockA{\textit{dept. name of organization (of Aff.)} \\
% \textit{name of organization (of Aff.)}\\
% City, Country \\
% email address or ORCID}
}

\maketitle

\begin{abstract}
    Caregivers of infants face considerable stress and anxiety around the safety and health
    of their babies, especially during sleep. Traditional baby monitors lack advanced health
   tracking capabilities and are primarily audio-based, which provides limited information
    on the baby’s well-being. The absence of comprehensive health monitoring often leaves
    caregivers unaware of potential health risks, such as fever, unusual heart rates, or dangerous
    sleeping postures that can increase the risk of Sudden Infant Death Syndrome (SIDS) or
    other breathing-related complications. Consequently, caregivers are forced to manually
    check on their babies frequently, disrupting their sleep and daily lives.
\end{abstract}

\begin{IEEEkeywords}
caregivers, infants, Sudden Infant Death Syndrome (SIDS), monitoring
\end{IEEEkeywords}

\section{Introduction}
The health and safety of infants are critical concerns for parents, par
ticularly when they are unable to provide constant supervision due to
 other responsibilities. One of the major risks to infants during sleep is
 Sudden Infant Death Syndrome (SIDS), which can occur if the baby un
knowingly assumes an unsafe sleeping posture. In addition to posture,
 environmental factors like temperature, humidity, and the baby’s health
 indicators—such as body temperature and heart rate—can have signifi
cant impacts on the baby’s well-being. The lack of real-time, compre
hensive monitoring systems makes it difficult for parents to detect these
 risks in time. This project, Cloud-Based Smart Monitoring System for
 Baby Health and Safety, is designed to bridge this gap by leveraging ad
vanced software algorithms and cloud-based solutions to provide real-time
 monitoring of a baby’s health and surroundings. With the integration of
 multiple sensors and a camera, the system ensures that any abnormalities,
 such as unsafe sleeping postures or sudden health changes, are detected
 and immediately communicated to the parents through a mobile applica
tion, helping to prevent potential health risks.
 With the advancement of technology, there has been a growing interest
 in creating smart monitoring systems that go beyond simple video surveil
lance, incorporating health data analytics. This project aims to build on
 existing systems by introducing an innovative, software-focused approach
 that can simultaneously monitor and process multiple parameters, such
 as the baby’s posture, heart rate, and environmental conditions. Using  cloud computing for real-time data processing and alerts, the system will
 allow parents to track their child’s well-being from any location, ensuring
 both the baby’s safety and the parents’ peace of mind. The focus on cloud
 infrastructure also allows scalability, enabling the system to be expanded
 with additional features and updates as needed.

 \subsection{Problem Statement}
To develop a cloud-based smart monitoring system that addresses the challenges parents face in continuously monitoring their infants, particularly when away from home. The system will use real-time data from sensors and video feeds to detect unsafe sleeping postures, abnormal body temperature, irregular heart rate, and environmental factors such as humidity. By using software-driven algorithms for analysis and alerting, the system will notify parents instantly of any concerns, thus preventing risks like Sudden Infant Death Syndrome (SIDS) and ensuring the infant’s health and safety.

\subsection{Objectives}
The objectives of the proposed project work are:
\begin{enumerate}
    \item To develop a mobile app that collects the body temperature of the baby and room temperature from the cloud, which is transmitted from the monitoring device.
    \item To integrate computer vision technology to detect unsafe sleeping positions of the baby.
    \item To create a user-friendly interface that allows parents to easily monitor real-time temperature readings regardless of the distance.
    \item To deliver actionable notifications through app alerts when abnormal readings or unsafe sleeping position is detected.
\end{enumerate}

\subsection{scope}
The \textbf{Cloud-Based Smart Monitoring System for Baby Health and Safety} aims to provide a comprehensive, software-driven solution for real-time monitoring of a baby’s health, environment, and movements. The project’s scope includes the development of advanced algorithms to detect unsafe sleeping postures using computer vision, as well as the integration of sensor data from temperature, humidity, and heart rate monitors. The software will process this data in real-time through a cloud infrastructure, delivering instant alerts to parents via a mobile application whenever abnormalities are detected, such as a sudden change in the baby’s sleeping position, body temperature, or crying. This monitoring will be continuous and remote, ensuring that parents receive timely notifications even when they are away from home.

The project is highly relevant in today’s fast-paced world, where parents are often unable to supervise their children around the clock. The system can be applied in homes, daycares, or hospitals, giving caregivers real-time insight into the baby’s well-being. By focusing on software for analyzing health and environmental data, this project addresses a significant gap in traditional baby monitors, which are often limited in functionality. The use of cloud technology ensures scalability, allowing for future enhancements such as the addition of more sensors or features, thereby making the system adaptable to evolving needs in infant care and monitoring. Regardless of the distance between the parent and the child, the vitals of the child can be monitored by the parents from any location.





\section{Literature Survey}

\subsection{IoT-Based Smart Baby Monitoring System with Emotion Recognition Using Machine Learning}
\textbf{Identified Problem:} Working parents face challenges in continuously monitoring their babies, particularly concerning environmental conditions and emotional states.

\textbf{Methodology:} An IoT-based system integrates sensors to monitor room temperature, humidity, and facial emotion recognition. Data is transmitted to the Blynk server for real-time monitoring via a mobile application.

\textbf{Implementation:} The system employs IoT sensors and machine learning algorithms to detect a baby’s cry and facial emotions. Notifications are sent to parents if abnormal conditions are detected.

\textbf{Results:} The implementation demonstrated effective monitoring capabilities, enabling parents to manage their time efficiently while ensuring their child’s well-being.

\textbf{Inference from Results:} The system alleviates parental burden by providing timely notifications and insights into the child’s emotional state.

\textbf{Limitations/Future Scope:} Further development is needed in data security, privacy, and improving emotion recognition accuracy.

\subsection{IoT-Based Baby Monitoring System}
\textbf{Identified Problem:} Developing a cost-effective and efficient real-time infant monitoring system.

\textbf{Methodology:} Utilizing NodeMCU as the control unit, integrating sensors for temperature, humidity, and crying detection. Data is uploaded to the Adafruit Blynk server for remote access.

\textbf{Implementation:} A prototype includes features like automatic cradle swaying upon detecting a baby’s cry and live video surveillance via an external webcam.

\textbf{Results:} The system proved effective in monitoring vital parameters while being simple and cost-efficient.

\textbf{Inference from Results:} The design ensures easy implementation and accessibility for many families.

\textbf{Limitations/Future Scope:} Future improvements should enhance sensor accuracy and expand functionalities to include additional health parameters.

\subsection{Internet of Things in Pregnancy Care Coordination and Management}
\textbf{Identified Problem:} Identifying gaps in existing literature regarding IoT applications in pregnancy and neonatal care.

\textbf{Methodology:} A systematic review of IoT systems in healthcare, focusing on monitoring pregnant women and newborns.

\textbf{Implementation:} The study synthesizes findings from various research papers to identify trends and challenges in IoT applications for maternal and infant health.

\textbf{Results:} IoT is increasingly important in healthcare, though challenges remain regarding data security and sensor accuracy.

\textbf{Inference from Results:} IoT has transformative potential, but critical gaps must be addressed for effective implementation.

\textbf{Limitations/Future Scope:} Future research should improve security protocols and enhance the user experience of IoT devices.

\subsection{Development of an IoT-Based Smart Baby Monitoring System with Face Recognition}
\textbf{Identified Problem:} Addressing parental anxiety regarding infant safety by proposing an advanced monitoring system.

\textbf{Methodology:} The system integrates face recognition technology with environmental monitoring sensors for comprehensive infant monitoring.

\textbf{Implementation:} Machine learning algorithms enable facial recognition alongside temperature and humidity monitoring.

\textbf{Results:} The solution showed high accuracy in recognizing faces and effectively monitored environmental conditions.

\textbf{Inference from Results:} This dual approach enhances parental confidence by providing real-time updates on the child’s identity and environmental safety.

\textbf{Limitations/Future Scope:} Challenges remain in ensuring robust performance under varying lighting conditions for facial recognition.

\subsection{IoT-Based Baby Monitoring System Smart Cradle}
\textbf{Identified Problem:} The need for automated baby care solutions for parents who cannot always be physically present.

\textbf{Methodology:} A smart cradle is designed using IoT technology to monitor crying, temperature, and humidity.

\textbf{Implementation:} A microcontroller automates actions like cradle swaying when a baby cries, integrating multiple sensors for monitoring.

\textbf{Results:} Testing confirmed effective environmental monitoring and automated responses to crying.

\textbf{Inference from Results:} The system reduces parental workload by automating basic caregiving functions.

\textbf{Limitations/Future Scope:} Future improvements could integrate advanced health monitoring features such as heart rate tracking.

\subsection{Smart Infant Baby Monitoring System Using IoT}
\textbf{Identified Problem:} Addressing Sudden Infant Death Syndrome (SIDS) through continuous health parameter monitoring.

\textbf{Methodology:} The system employs Raspberry Pi and sensors to track temperature, heart rate, and sound detection.

\textbf{Implementation:} Data is transmitted via SMS notifications to parents upon detecting abnormalities, ensuring ease of use.

\textbf{Results:} The system significantly reduced SIDS risk and provided high parental satisfaction due to its reliability.

\textbf{Inference from Results:} Remote monitoring enhances infant safety and reduces parental anxiety.

\textbf{Limitations/Future Scope:} Future research should integrate predictive analytics for early health issue detection.

\subsection{Development of RTO-Based Internet-Connected Baby Monitoring System}
\textbf{Identified Problem:} The lack of real-time access to critical infant health metrics due to fragmented monitoring systems.

\textbf{Methodology:} The proposed system uses various sensors to track temperature, humidity, and baby motion.

\textbf{Implementation:} Sensor data is stored in a cloud database and accessible via a user-friendly mobile application with real-time alerts.

\textbf{Results:} The system demonstrated reliable data transmission and effective alerts for abnormal readings, improving parental engagement.

\textbf{Inference from Results:} Continuous access to essential health metrics empowers parents to take timely interventions.

\textbf{Limitations/Future Scope:} Future research should enhance the user interface and integrate additional sensors for more complex health indicators.

\subsection{Smart Caregiving Support Cloud Integration Systems}
\textbf{Identified Problem:} Current baby monitoring solutions operate independently, leading to fragmented user experiences.

\textbf{Methodology:} The proposed system integrates cloud computing technologies to consolidate multiple sensor outputs into a single platform.

\textbf{Implementation:} Advanced cloud technologies aggregate data from temperature monitors and motion detectors, generating alerts when abnormalities occur.

\textbf{Results:} Better data synchronization improved parental response times in emergencies, demonstrating the effectiveness of integration over isolated systems.

\textbf{Inference from Results:} Integrated systems provide holistic insights, improving caregiver decision-making regarding child safety and well-being.

\textbf{Limitations/Future Scope:} Future research should explore scalability and further device integrations to enhance user experience.

\subsection{Real-Time Infant Health Monitoring System for Hard of Hearing Parents}
\textbf{Identified Problem:} Traditional monitoring methods lack immediacy and accessibility for hard-of-hearing parents.

\textbf{Methodology:} The system employs IoT technologies to capture vital signs and environmental conditions in real-time.

\textbf{Implementation:} Sensor data is processed in real-time and made accessible through an intuitive mobile interface.

\textbf{Results:} The prototype effectively provided continuous updates, enabling quick parental intervention when necessary.

\textbf{Inference from Results:} Real-time insights empower parents with critical knowledge, significantly enhancing child safety measures.

\textbf{Limitations/Future Scope:} Future work should explore integration with healthcare providers for a more comprehensive support system.






\section{Prepare Your Paper Before Styling}
Before you begin to format your paper, first write and save the content as a 
separate text file. Complete all content and organizational editing before 
formatting. Please note sections \ref{AA} to \ref{FAT} below for more information on 
proofreading, spelling and grammar.

Keep your text and graphic files separate until after the text has been 
formatted and styled. Do not number text heads---{\LaTeX} will do that 
for you.

\subsection{Abbreviations and Acronyms}\label{AA}
Define abbreviations and acronyms the first time they are used in the text, 
even after they have been defined in the abstract. Abbreviations such as 
IEEE, SI, MKS, CGS, ac, dc, and rms do not have to be defined. Do not use 
abbreviations in the title or heads unless they are unavoidable.

\subsection{Units}
\begin{itemize}
\item Use either SI (MKS) or CGS as primary units. (SI units are encouraged.) English units may be used as secondary units (in parentheses). An exception would be the use of English units as identifiers in trade, such as ``3.5-inch disk drive''.
\item Avoid combining SI and CGS units, such as current in amperes and magnetic field in oersteds. This often leads to confusion because equations do not balance dimensionally. If you must use mixed units, clearly state the units for each quantity that you use in an equation.
\item Do not mix complete spellings and abbreviations of units: ``Wb/m\textsuperscript{2}'' or ``webers per square meter'', not ``webers/m\textsuperscript{2}''. Spell out units when they appear in text: ``. . . a few henries'', not ``. . . a few H''.
\item Use a zero before decimal points: ``0.25'', not ``.25''. Use ``cm\textsuperscript{3}'', not ``cc''.)
\end{itemize}

\subsection{Equations}
Number equations consecutively. To make your 
equations more compact, you may use the solidus (~/~), the exp function, or 
appropriate exponents. Italicize Roman symbols for quantities and variables, 
but not Greek symbols. Use a long dash rather than a hyphen for a minus 
sign. Punctuate equations with commas or periods when they are part of a 
sentence, as in:
\begin{equation}
a+b=\gamma\label{eq}
\end{equation}

Be sure that the 
symbols in your equation have been defined before or immediately following 
the equation. Use ``\eqref{eq}'', not ``Eq.~\eqref{eq}'' or ``equation \eqref{eq}'', except at 
the beginning of a sentence: ``Equation \eqref{eq} is . . .''

\subsection{\LaTeX-Specific Advice}

Please use ``soft'' (e.g., \verb|\eqref{Eq}|) cross references instead
of ``hard'' references (e.g., \verb|(1)|). That will make it possible
to combine sections, add equations, or change the order of figures or
citations without having to go through the file line by line.

Please don't use the \verb|{eqnarray}| equation environment. Use
\verb|{align}| or \verb|{IEEEeqnarray}| instead. The \verb|{eqnarray}|
environment leaves unsightly spaces around relation symbols.

Please note that the \verb|{subequations}| environment in {\LaTeX}
will increment the main equation counter even when there are no
equation numbers displayed. If you forget that, you might write an
article in which the equation numbers skip from (17) to (20), causing
the copy editors to wonder if you've discovered a new method of
counting.

{\BibTeX} does not work by magic. It doesn't get the bibliographic
data from thin air but from .bib files. If you use {\BibTeX} to produce a
bibliography you must send the .bib files. 

{\LaTeX} can't read your mind. If you assign the same label to a
subsubsection and a table, you might find that Table I has been cross
referenced as Table IV-B3. 

{\LaTeX} does not have precognitive abilities. If you put a
\verb|\label| command before the command that updates the counter it's
supposed to be using, the label will pick up the last counter to be
cross referenced instead. In particular, a \verb|\label| command
should not go before the caption of a figure or a table.

Do not use \verb|\nonumber| inside the \verb|{array}| environment. It
will not stop equation numbers inside \verb|{array}| (there won't be
any anyway) and it might stop a wanted equation number in the
surrounding equation.

\subsection{Some Common Mistakes}\label{SCM}
\begin{itemize}
\item The word ``data'' is plural, not singular.
\item The subscript for the permeability of vacuum $\mu_{0}$, and other common scientific constants, is zero with subscript formatting, not a lowercase letter ``o''.
\item In American English, commas, semicolons, periods, question and exclamation marks are located within quotation marks only when a complete thought or name is cited, such as a title or full quotation. When quotation marks are used, instead of a bold or italic typeface, to highlight a word or phrase, punctuation should appear outside of the quotation marks. A parenthetical phrase or statement at the end of a sentence is punctuated outside of the closing parenthesis (like this). (A parenthetical sentence is punctuated within the parentheses.)
\item A graph within a graph is an ``inset'', not an ``insert''. The word alternatively is preferred to the word ``alternately'' (unless you really mean something that alternates).
\item Do not use the word ``essentially'' to mean ``approximately'' or ``effectively''.
\item In your paper title, if the words ``that uses'' can accurately replace the word ``using'', capitalize the ``u''; if not, keep using lower-cased.
\item Be aware of the different meanings of the homophones ``affect'' and ``effect'', ``complement'' and ``compliment'', ``discreet'' and ``discrete'', ``principal'' and ``principle''.
\item Do not confuse ``imply'' and ``infer''.
\item The prefix ``non'' is not a word; it should be joined to the word it modifies, usually without a hyphen.
\item There is no period after the ``et'' in the Latin abbreviation ``et al.''.
\item The abbreviation ``i.e.'' means ``that is'', and the abbreviation ``e.g.'' means ``for example''.
\end{itemize}
An excellent style manual for science writers is \cite{b7}.

\subsection{Authors and Affiliations}\label{AAA}
\textbf{The class file is designed for, but not limited to, six authors.} A 
minimum of one author is required for all conference articles. Author names 
should be listed starting from left to right and then moving down to the 
next line. This is the author sequence that will be used in future citations 
and by indexing services. Names should not be listed in columns nor group by 
affiliation. Please keep your affiliations as succinct as possible (for 
example, do not differentiate among departments of the same organization).

\subsection{Identify the Headings}\label{ITH}
Headings, or heads, are organizational devices that guide the reader through 
your paper. There are two types: component heads and text heads.

Component heads identify the different components of your paper and are not 
topically subordinate to each other. Examples include Acknowledgments and 
References and, for these, the correct style to use is ``Heading 5''. Use 
``figure caption'' for your Figure captions, and ``table head'' for your 
table title. Run-in heads, such as ``Abstract'', will require you to apply a 
style (in this case, italic) in addition to the style provided by the drop 
down menu to differentiate the head from the text.

Text heads organize the topics on a relational, hierarchical basis. For 
example, the paper title is the primary text head because all subsequent 
material relates and elaborates on this one topic. If there are two or more 
sub-topics, the next level head (uppercase Roman numerals) should be used 
and, conversely, if there are not at least two sub-topics, then no subheads 
should be introduced.

\subsection{Figures and Tables}\label{FAT}
\paragraph{Positioning Figures and Tables} Place figures and tables at the top and 
bottom of columns. Avoid placing them in the middle of columns. Large 
figures and tables may span across both columns. Figure captions should be 
below the figures; table heads should appear above the tables. Insert 
figures and tables after they are cited in the text. Use the abbreviation 
``Fig.~\ref{fig}'', even at the beginning of a sentence.

\begin{table}[htbp]
\caption{Table Type Styles}
\begin{center}
\begin{tabular}{|c|c|c|c|}
\hline
\textbf{Table}&\multicolumn{3}{|c|}{\textbf{Table Column Head}} \\
\cline{2-4} 
\textbf{Head} & \textbf{\textit{Table column subhead}}& \textbf{\textit{Subhead}}& \textbf{\textit{Subhead}} \\
\hline
copy& More table copy$^{\mathrm{a}}$& &  \\
\hline
\multicolumn{4}{l}{$^{\mathrm{a}}$Sample of a Table footnote.}
\end{tabular}
\label{tab1}
\end{center}
\end{table}

\begin{figure}[htbp]
\centerline{\includegraphics{fig1.png}}
\caption{Example of a figure caption.}
\label{fig}
\end{figure}

Figure Labels: Use 8 point Times New Roman for Figure labels. Use words 
rather than symbols or abbreviations when writing Figure axis labels to 
avoid confusing the reader. As an example, write the quantity 
``Magnetization'', or ``Magnetization, M'', not just ``M''. If including 
units in the label, present them within parentheses. Do not label axes only 
with units. In the example, write ``Magnetization (A/m)'' or ``Magnetization 
\{A[m(1)]\}'', not just ``A/m''. Do not label axes with a ratio of 
quantities and units. For example, write ``Temperature (K)'', not 
``Temperature/K''.

\section*{Acknowledgment}

The preferred spelling of the word ``acknowledgment'' in America is without 
an ``e'' after the ``g''. Avoid the stilted expression ``one of us (R. B. 
G.) thanks $\ldots$''. Instead, try ``R. B. G. thanks$\ldots$''. Put sponsor 
acknowledgments in the unnumbered footnote on the first page.

\section*{References}

Please number citations consecutively within brackets \cite{b1}. The 
sentence punctuation follows the bracket \cite{b2}. Refer simply to the reference 
number, as in \cite{b3}---do not use ``Ref. \cite{b3}'' or ``reference \cite{b3}'' except at 
the beginning of a sentence: ``Reference \cite{b3} was the first $\ldots$''

Number footnotes separately in superscripts. Place the actual footnote at 
the bottom of the column in which it was cited. Do not put footnotes in the 
abstract or reference list. Use letters for table footnotes.

Unless there are six authors or more give all authors' names; do not use 
``et al.''. Papers that have not been published, even if they have been 
submitted for publication, should be cited as ``unpublished'' \cite{b4}. Papers 
that have been accepted for publication should be cited as ``in press'' \cite{b5}. 
Capitalize only the first word in a paper title, except for proper nouns and 
element symbols.

For papers published in translation journals, please give the English 
citation first, followed by the original foreign-language citation \cite{b6}.

\begin{thebibliography}{00}
\bibitem{b1} G. Eason, B. Noble, and I. N. Sneddon, ``On certain integrals of Lipschitz-Hankel type involving products of Bessel functions,'' Phil. Trans. Roy. Soc. London, vol. A247, pp. 529--551, April 1955.
\bibitem{b2} J. Clerk Maxwell, A Treatise on Electricity and Magnetism, 3rd ed., vol. 2. Oxford: Clarendon, 1892, pp.68--73.
\bibitem{b3} I. S. Jacobs and C. P. Bean, ``Fine particles, thin films and exchange anisotropy,'' in Magnetism, vol. III, G. T. Rado and H. Suhl, Eds. New York: Academic, 1963, pp. 271--350.
\bibitem{b4} K. Elissa, ``Title of paper if known,'' unpublished.
\bibitem{b5} R. Nicole, ``Title of paper with only first word capitalized,'' J. Name Stand. Abbrev., in press.
\bibitem{b6} Y. Yorozu, M. Hirano, K. Oka, and Y. Tagawa, ``Electron spectroscopy studies on magneto-optical media and plastic substrate interface,'' IEEE Transl. J. Magn. Japan, vol. 2, pp. 740--741, August 1987 [Digests 9th Annual Conf. Magnetics Japan, p. 301, 1982].
\bibitem{b7} M. Young, The Technical Writer's Handbook. Mill Valley, CA: University Science, 1989.
\bibitem{b8} D. P. Kingma and M. Welling, ``Auto-encoding variational Bayes,'' 2013, arXiv:1312.6114. [Online]. Available: https://arxiv.org/abs/1312.6114
\bibitem{b9} S. Liu, ``Wi-Fi Energy Detection Testbed (12MTC),'' 2023, gitHub repository. [Online]. Available: https://github.com/liustone99/Wi-Fi-Energy-Detection-Testbed-12MTC
\bibitem{b10} ``Treatment episode data set: discharges (TEDS-D): concatenated, 2006 to 2009.'' U.S. Department of Health and Human Services, Substance Abuse and Mental Health Services Administration, Office of Applied Studies, August, 2013, DOI:10.3886/ICPSR30122.v2
\bibitem{b11} K. Eves and J. Valasek, ``Adaptive control for singularly perturbed systems examples,'' Code Ocean, Aug. 2023. [Online]. Available: https://codeocean.com/capsule/4989235/tree
\end{thebibliography}

\vspace{12pt}
\color{red}
IEEE conference templates contain guidance text for composing and formatting conference papers. Please ensure that all template text is removed from your conference paper prior to submission to the conference. Failure to remove the template text from your paper may result in your paper not being published.

\end{document}
