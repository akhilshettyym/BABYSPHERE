\documentclass[12pt,a4paper]{report}
\usepackage[utf8]{inputenc}
\usepackage{amsfonts}
\usepackage{setspace}
\usepackage{graphicx}
\usepackage{array}
\usepackage{fancyhdr}
\usepackage{geometry}
\usepackage{ragged2e}
\usepackage{color}
\usepackage{biblatex}
\addbibresource{pic/reference.bib}

\geometry{
a4paper,
total={210mm,297mm},
left=1.0in,
right=0.85in,
top=0.75in,
bottom=0.75in,
}
\begin{document}
\pagestyle{empty}
\begin{center}

{\large \textbf{Visvesvaraya Technological University, Belagavi – 590018}}
\begin{figure}[hbtp]
\centering
\includegraphics[width=2.3cm,height=3cm]{./pic/vtu}
\end{figure}

\textbf{PROJECT PROPOSAL}
\par
\textbf{ON}
\par
\vspace{6pt}
{\Large \textbf{Cloud-Based Smart Monitoring System for Baby Health and Safety}}
\par
\vspace{12pt}
\par
\textit{\textbf{Submitted in partial fulfillment of the requirements for the degree }}
\par
\vspace{12pt}
\large \textbf{BACHELOR OF ENGINEERING }
\par
\textbf{in}
\par
\large \textbf{COMPUTER SCIENCE \& ENGINEERING}
\par
\vspace{12pt}
\textit{\textbf{Submitted by}}
\vspace{8pt}

\begin{center}
\begin{tabular}{l@{\hspace{2cm}}r}
\textbf{\large Aaron Tauro} & \textbf{4SO21CS002} \\
\textbf{\large Abhik L Salian} & \textbf{4SO21CS004} \\
\textbf{\large Akhil Shetty M} & \textbf{4SO21CS013} \\
\textbf{\large H Karthik P Nayak} & \textbf{4SO21CS058} \\
\end{tabular}
\end{center}

\vspace{12pt}
\textit{\textbf{Under the Guidance of}}
\par
\vspace{6pt}
\textbf{Dr Sridevi Saralaya}
\par
\vspace{2pt}
\normalsize { Professor, Department of CSE }
\par
\begin{figure}[hbtp]
\centering
\includegraphics[scale=0.6]{./pic/sjeclogo}
\end{figure}
\large \textbf{DEPT. OF COMPUTER SCIENCE AND ENGINEERING}
\par \Large \textbf{ST JOSEPH ENGINEERING COLLEGE}
\par 
\textbf{An Autonomous Institution}
\par
{\large{(Affiliated to VTU Belagavi, Recognized by AICTE, Accredited by NBA)}}
\par
\vspace{3pt}
{\large \textbf{Vamanjoor, Mangaluru - 575028, Karnataka}}
\par 
\vspace{12pt}
{\Large \textbf{2024-25}}
\end{center}
\newpage

%------ Body of document ----------------------
\pagestyle{plain}
\setstretch{1.5}
\pagenumbering{roman}

\section*{Project Title}
Cloud-Based Smart Monitoring System for Baby Health and Safety

\section*{Type of Project}
Product based project.

\section*{Background for the  the project}
In recent years, the rise in Sudden Infant Death Syndrome (SIDS)
 and other health concerns related to newborns has highlighted 
 the need for continuous and accurate monitoring of babies\cite{1}. 
 Many parents face the challenge of balancing their busy 
 schedules while ensuring their child’s safety, especially when 
 they are not physically present. Traditional monitoring devices
  often lack intelligent features that can detect critical signs,
   such as alerting the parent about variations in body temperature or unsafe sleeping positions. With 
   advancements in cloud technology and computer vision, there 
   is an opportunity to develop smart monitoring systems that 
   provide real-time alerts and actionable insights. Our project aims to address these issues by 
   creating a cloud-based system capable of monitoring baby 
   body temperature, room conditions, and movements using 
   computer vision, helping parents prevent potential dangers 
   like SIDS through timely alerts.

\section*{Objectives }
The objectives of the proposed project work are:
\begin{enumerate}
    \item To develop a mobile app that collects the body temperature of the baby and room temperature from the cloud, which is transmitted from the monitoring device.
    \item To integrate computer vision technology to detect unsafe sleeping positions of the baby.
    \item To create a user-friendly interface that allows parents to easily monitor real-time temperature readings.
    \item To deliver actionable notifications through app alerts when abnormal readings or unsafe sleeping position is detected.
\end{enumerate}

\section*{Software / Hardware Requirements}
\subsection*{Software Requirements}
\begin{enumerate}
    \item React Native development framework to build the mobile app for cross-platform support\cite{2}.
    \item Firebase cloud platform for real-time database management, data storage, and authentication\cite{3}.
    \item TensorFlow Lite or OpenCV libraries for implementing computer vision algorithms on mobile devices to detect unsafe sleeping positions\cite{4}\cite{5}\cite{6}.
    \item Mobile-optimized signal processing algorithms to analyze camera data for non-contact heart rate monitoring\cite{7}\cite{8}\cite{9}.
    \item Node.js or similar backend technologies for handling API requests and interactions with cloud services.
    \item Firebase Cloud Messaging or similar services for sending push notifications related to temperature changes, movement detection, or abnormal heart rate.
    \item WebSockets or Firebase Realtime Database to enable real-time data updates in the app for temperature, movement, and heart rate monitoring.
    \item Firebase Authentication to manage user accounts and secure access to baby monitoring data.
\end{enumerate}
\subsection*{Hardware Requirements}
\begin{enumerate}
    \item A computer or a laptop with operating system Windows 10 or higher (64-bit), and macOS Big Sur or higher (for iOS app development, macOS is required).
    \item Intel Core i5 (7th generation or higher) or AMD Ryzen 5 or better processor.
    \item Integrated GPU such as Radeon or NVIDIA GTX should work fine rather than opting for dedicated GPUs.
    \item Minimum 8 GB RAM is required.
    \item Hard disk with a minimum available space of 50 GB.
    \item A reliable internet connection is essential for downloading packages, accessing cloud services (e.g., Firebase), and for real-time data transmission.
    \item A mobile device running on Android 10 or higher, and another device running on iOS (if implementing for iOS) for testing of the app.
\end{enumerate}
\newpage
\pagestyle{plain}
\renewcommand{\bibname}{References}
\addcontentsline{toc}{chapter}{References}

% \printbibliography
\begin{thebibliography}{35}
\bibitem{1}
Moon, Rachel Y., Rosemary SC Horne, and Fern R. Hauck. ``Sudden infant death syndrome." The Lancet 370, no. 9598 (2007): 1578-1587.
\bibitem{2}

React Native, ``React Native Documentation," 2023. [Online]. Available: https://reactnative.dev/docs/getting-started.

\bibitem{3}
Firebase, ``Firebase Documentation," 2023. [Online]. Available: https://firebase.google.com/docs.


\bibitem{4}
Okuno, Ayaka, Takaaki Ishikawa, and Hiroshi Watanabe. ``Rollover detection of infants using posture estimation model." In 2020 IEEE 9th Global Conference on Consumer Electronics (GCCE), pp. 490-493. IEEE, 2020.

\bibitem{5}
Raghavan, Neethu, and S. Ullas. ``Infant movement detection and constant monitoring using wireless sensors." In 2017 International Conference on Wireless Communications, Signal Processing and Networking (WiSPNET), pp. 2109-2114. IEEE, 2017.

\bibitem{6}
Singh, Gurpreet, Abhishek Raj Shekhar, Xinrui Yu, and Jafar Saniie. ``Smart Infant Monitoring System Using Computer Vision and AI." In 2023 IEEE International Conference on Electro Information Technology (eIT), pp. 1-6. IEEE, 2023.

\bibitem{7}
Chen, Yenming J., Lung-Chang Lin, Shu-Ting Yang, Kao-Shing Hwang, Chia-Te Liao, and Wen-Hsien Ho. ``High-reliability non-contact photoplethysmography imaging for newborn care by a generative artificial intelligence." IEEE Access (2023).

\bibitem{8}
Tang, Chuanxiang, Jiwu Lu, and Jie Liu. ``Non-contact heart rate monitoring by combining convolutional neural network skin detection and remote photoplethysmography via a low-cost camera." In Proceedings of the IEEE Conference on Computer Vision and Pattern Recognition Workshops, pp. 1309-1315. 2018.

\bibitem{9}
Tran, Duc Nhan, Hyukzae Lee, and Changick Kim. ``A robust real time system for remote heart rate measurement via camera." In 2015 IEEE International Conference on Multimedia and Expo (ICME), pp. 1-6. IEEE, 2015.

\end{thebibliography}

\end{document}